\documentclass[a4paper,11pt]{article}

\usepackage{wrapfig}
\usepackage{amsmath}
\usepackage{pgfplots}
\usepackage{xcolor}
\usepackage[most]{tcolorbox}
%Russian-specific packages
%--------------------------------------
\usepackage[T2A]{fontenc}
\usepackage[utf8]{inputenc}
\usepackage[russian, english]{babel}
%--------------------------------------

\definecolor{lemonchiffon}{rgb}{1.0, 0.98, 0.8}
\newtcolorbox{mainblock}{colback=lemonchiffon,grow to right by=-10mm,grow to left by=-10mm,
boxrule=0pt,boxsep=0pt,breakable} % настройки области с изменённым фоном
\definecolor{block-gray}{gray}{0.95} % уровень прозрачности (1 - максимум)
\newtcolorbox{importantblock}{colback=block-gray,grow to right by=-10mm,grow to left by=-10mm,
boxrule=0pt,boxsep=0pt,breakable} % настройки области с изменённым фоном

\makeatletter
\newcommand{\settag}[1]{
  \tag*{(#1)\qquad}
  \edef\@currentlabel{\theequation#1}}
\makeatother

\title{29. Метод стрельбы для решения краевых задач. Сведение дифференциального уравнения выского порядка
      к системе уравенний первого порядка}
\author{Андрей Бареков \and Ярослав Пылаев \and По лекциям Устинова С.М.}
\date{\today}

\begin{document}
\maketitle
\newpage

\section{Методы решения краевых задач для дифференциальных уравений}
\begin{equation*}
  \frac{dx}{dt} = f(t, x), \hspace{5mm} t \in [a, b].
\end{equation*}
В \textit{задаче Коши} все начальные условия задаются в точке $a$. В общем случае они могут задаваться в любой точке
на промежутке $[a, b]$, чаще всего их задают на концах (краях) промежутка - такие задачи называются \textit{краевые}. \\

\noindent Методы решения краевых задач делится на две группы, когда исходная задача сводится к
\begin{itemize}
  \item многократному решению зачачи Коши;
  \item решению систем линейных и нелинейных алгебраических уравнений.
\end{itemize}

\noindent Типичным представителем первой группы является \textit{метод стрельбы}.

\subsection{Метод стрельбы (пристрелки)}
Проиллюстрируем на примере уравнения 2-го порядка:
\begin{align*}
  x = \begin{pmatrix} U \\ V \end{pmatrix}, && \frac{dU}{dt} = f_1(t, U, V), && \frac{dV}{dt} = f_2(t, U, V), && t \in [a, b],
\end{align*}
\begin{equation*}
  U(a)=U_a, V(b)=V_b.
\end{equation*}


\end{document}
