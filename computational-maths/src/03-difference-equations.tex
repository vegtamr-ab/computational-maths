\documentclass[a4paper,11pt]{article}

\usepackage{wrapfig}
\usepackage{amsmath}
\usepackage{pgfplots}
\usepackage[most]{tcolorbox} % для управления цветом
\usepackage{chngcntr}
%Russian-specific packages
%--------------------------------------
\usepackage[T2A]{fontenc}
\usepackage[utf8]{inputenc}
\usepackage[russian, english]{babel}
%--------------------------------------

%Для выделения блока текста в рамку
\definecolor{block-gray}{gray}{0.95} % уровень прозрачности (1 - максимум)
\newtcolorbox{importantblock}{colback=block-gray,grow to right by=-10mm,grow to left by=-10mm,
boxrule=0pt,boxsep=0pt,breakable} % настройки области с изменённым фоном

\counterwithin*{equation}{section}% Сбрасывает счетчик формул в \section

\title{03. Разностное уравнение, его порядок. Линейные разностные уравнения первого порядка и порядка выше первого.}
\author{Андрей Бареков \and Ярослав Пылаев \and По лекциям Устинова С.М.}
\date{\today}

\begin{document}
\maketitle
\newpage

\section{Разностное уравнение. Его порядок}
  Первоначально остановимся на дифференциальном уравнении. Дифференциальным уравнением называется уравнение, которое связывает исходную функцию и ее проиводную.
  \begin{equation}
  y^{s}(t) = F(t, y(t), y'(t), y''(t), ..., y^{s-1}(t)) \\
  \end{equation}
  \begin{center}
  Разрешенное относительно старших производных
  \end{center}

  \begin{itemize}
    \item \textbf{Порядок дифференциального уравнения} определяется порядком старшей производной
    \item Порядок уравнения определяет \textit{количество начальных условий}, необходимых для однозначного решения задачи, а если уравнеие является линейным относительно функции $y(t)$ и ее производных, то - \textit{количество линейно независимых решений}.
  \end{itemize}

  По аналогии можно было бы ввести понятие \textbf{порядка разностного уравнения}
  \begin{equation}
    2\Delta^3 x_n + 3\Delta^2 x_n - x_n = 0
    \label{eq:DiffEq}
  \end{equation}

  Логично ожидать, что уравнение \ref{eq:DiffEq} имеет 3-й порядок, \textit{однако}
  \marginpar
  {
    \footnotesize Из вывода конечных разностей высшего порядка
  }
  \begin{gather*}
    2(x_{n+3} - 3x_{n+2} + 3x_{n+1} - x_n) + 3(x_{n+2} - 2x_{n+1} + x_n) - x_n = 0 \\
    2x_{n+3} - 3x_{n+2} = 0 \Rightarrow x_{n+3} = 1.5x_{n+2}
  \end{gather*}
  \begin{center}
    Порядок этого уравнения: $ n+3 - (n+2) = 1$
  \end{center}
  \begin{importantblock}
    Для определения порядка разностных уравений необходио выразить все конечные разности через значения функций и привести подобные. Тогда разность между наибольшим и наименьшим значением аргумента определяет порядок разностного уравнения.
  \end{importantblock}
  \begin{gather*}
    F(k, f(k), f(k+1),..., f(k+s)) = 0 \\
    f(k+s)=F(k, f(k),..., f(k+s-1))
  \end{gather*}
  \begin{center}
    \small Второе уравнение в виде, разреш. отн. фу-ии с наибольшим знач. арг.\\
  \end{center}
  Порядок обоих уравений $s=(k+s)-k$, таким образом для их однозначного решения необходимо задать $s$ начальных условий $f(0),f(1),...f(s-1)$

  \subsection{Пошаговый метод решения разностного уравнения}
  Рассмотрим общий вид разностного уравнения, разрешенного относительно функции с наибольшим аргументом
  \begin{equation}
    x_{n+s} = F(n, x_{n+s-1},x_{n+s-2},..., x_n)
    \label{eq:DiffEqSpecView}
  \end{equation}
  \begin{center}
    Это уравнение порядка $s$ $\Rightarrow$ необходимо задать \\
    $s$ начальных условий $ x_0, x_1,..., x_{s-1} $
  \end{center}
  Полагаем в уравнении \ref{eq:DiffEqSpecView} $n=0, 1, 2, ...$
  \begin{gather*}
    x_{0+s} = F(0, x_{s-1},x_{s-2},..., x_0) \\
    x_{s+1} = F(1, x_s,x_{s-1},..., x_1) \\
    x_{s+2} = ...
  \end{gather*}
  Пошаговый метод совместно с начальными условиями всегда дает решение и обеспечивает существование и единственность решения начальной задачи для разностного уравнения.\\

  В некоторых случаях возможно получить выражение $x(n)$ в явном виде, \textit{не используя пошаговый метод}.

\section{Линейное разностное ур-ие 1-го порядка}
  \marginpar
  {
    \footnotesize
    {
      $\alpha$ - const, \\
      $\varphi(n)$ - ф-ия, \\
      $x_0$ - нач.усл
    }
  }
  \begin{equation}
    x_{n+1}=\alpha x_n + \varphi(n)
    \label{eq:LineDiffEq}
  \end{equation}
  Применим пошаговый метод:
  \begin{gather*}
    x_1 = \alpha x_0 + \varphi_0 \\
    x_2 = \alpha x_1 + \varphi_1 = \alpha^2 x_0 + \alpha \varphi_0 + \varphi_1 \\
    x_3 = \alpha x_2 + \varphi_2 = \alpha^3 x_0 + \alpha^2 \varphi_0 + \alpha \varphi_1 + \varphi_2 \\
    ...
  \end{gather*}
  \begin{importantblock}
    По индукции можно показать, что
    \begin{equation}
      x_n = \alpha^n x_0 + \sum_{k=0}^{n-1} \alpha^k \varphi_{n-1-k}
      \end{equation}
    точное решение уравнения (\ref{eq:LineDiffEq}).
  \end{importantblock}
  \marginpar
  {
    \small{$\sum_{k=0}^{n-1} \alpha^k \\ = \frac{1-\alpha^n}{1-\alpha}$ \\}
    \footnotesize{См. \\ "02. Сумм. по частям"}
  }
  Рассмотрим случай, когда функция $\varphi(n)=\beta$ - $const$
  \begin{equation}
    x_n = \alpha^n x_0 + \bigg(\sum_{k=0}^{n-1}\alpha^k \bigg) \beta = \alpha^n x_0 + \frac{1-\alpha^n}{1-\alpha} \beta
  \end{equation}

\section{Линейное разностное ур-ие порядка \textit{выше 1-го}}
  \begin{equation}
    \alpha_0 x_{n+s} + \alpha_1 x_{n+s-1} +...+ \alpha_s x_n = \varphi(n)
    \label{eq:LineDiffEqHighOrder}
  \end{equation}
  \begin{center}
    $x_0, x_1, ..., x_{s-1}$ - $s$ начальных условий
  \end{center}
  Линейное \textit{неоднородное} уравнение порядка $s$ с постоянными коэффициентами.
  \begin{importantblock}
    Уравнение (\ref{eq:LineDiffEqHighOrder}) называется \textit{однородным}, если правая часть тождественно равна $0$ и \textit{неоднородным} в противном случае.
  \end{importantblock}

  \subsection{Решение}
    Решение уравения \ref{eq:LineDiffEqHighOrder} начинаем с решения \textit{однородного} уравнения
    \begin{equation}
      \alpha_0 x_{n+s} + \alpha_1 x_{n+s-1} +...+ \alpha_s x_n = 0
    \end{equation}

    Будем искать \textbf{решение в виде}:
    \begin{equation}
    \Large {
      \boxed{x_n = C\gamma^n}
    }
    \end{equation}
%I'm sorry but it's too boring to name formulars labels and then I write its numbers like in lecture material
    Подставим (3) в (2) и сократим на $C \gamma^n$
    \begin{equation}
      \boxed{\alpha_0 \gamma^s + \alpha_1 \gamma^{s-1} +...+ \alpha_s = 0}
    \end{equation}
    \begin{center}
      \small{\textit{Характеристическое уравнение} для уравнений (1) и (2), а его корни $\gamma_1, \gamma_2,..., \gamma_s$- характеристические корни.}
    \end{center}
    Первоначально положим, что все корни уравнения (4) различны \boxed{i}. \\

    Пусть $x^*_n$, $x_n^{**}$ - 2 любых решения уравнения (2). Тогда любая их линенйная комбинация $(D_1x_n^* + D_2x_n^{**})$ также является решением \\ уравнения (2). В этом \underline{можно убедиться}, подставив эту линейную комбинацию в уравнение (2) и сгруппировав однородные слогаемые: \\
    \small
    {
      \begin{gather*}
        \begin{split}
          \alpha_0 x_{n+s}^* + \alpha_1 x_{n+s-1}^* +...+ \alpha_s x_n^* = 0\\
          \alpha_0 x_{n+s}^{**} + \alpha_1 x_{n+s-1}^{**} +...+ \alpha_s x_n^{**} = 0
          \end{split}
      \end{gather*}
      Тогда
      \begin{gather*}
        \alpha_0 (D_1x_{n+s}^*+D_2x_{n+s}^{**}) + \alpha_1 (D_1x_{n+s-1}^*+D_2x_{n+s}^{**-1}) +...+
        \alpha_s (D_1x_{n}^*+D_2x_{n}^{**}) \stackrel{?}{=} 0 \\
        D_1(\alpha_0 x_{n+s}^* + \alpha_1 x_{n+s-1}^* +...+ \alpha_s x_n^*) +
        D_2(\alpha_0 x_{n+s}^{**} + \alpha_1 x_{n+s-1}^{**} +...+ \alpha_s x_n^{**}) \stackrel{?}{=} 0
      \end{gather*}
      Выражения в скобках равны $0$ по условию, поэтому $D_1\times0 + D_2\times0 = 0 + 0 = 0$
      \begin{center}
        $\Rightarrow$ $(D_1x_n^* + D_2x_n^{**})$ является решением уравения (2).
      \end{center}
    }

    Поэтому общее решение уравнения (2) имеет вид:
    \begin{equation}
      x_n = \sum_{k=1}^{s}C_k\gamma_k^n
    \end{equation}
    \begin{center}
      Решение однородного уравнения \\
    \end{center}

    Перейдем к неоднородному уравнению. \\

    Пусть $\widetilde{x_n}$ - любое частное решение уравнения (1) (необязательно совпадающее с начальными условиями). \\
    Тогда \textit{общее решние неоднородного уравнения} имеет вид:
    \begin{equation}
      \Large
      {
        \boxed{x_n = \widetilde{x_n} + \sum_{k=1}^{s}C_k\gamma_k^n} \\
      }
    \end{equation}

    Далее для нахождения ${C_k}$ запишем формулу (6) для $n=0,1,...,s-1$, используя начальные условия, получив линейную систему из $s$ уравнений относительно $s$ неизвестных коэффициентов $C_k$. \\
    \begin{itemize}
      \item Если же в правой части исходного уравнения стоит линейная комбинация полиномов, экспонент, синусов и косинусов, то $\widetilde{x_n}$ подбирается в том же виде.
      \item В общем случае для поиска $\widetilde{x_n}$ используется :\\
        $\textbf{метод Лагранжа в вариации произвольных коэффициентов}$.
    \end{itemize}

    \vspace*{20mm}
    \hrule
    \boxed{i} Если в уравнении (4) есть кратные корни, например, $\gamma_1 \rightarrow r$ (встречается r раз).
    Тогда в формулах (5) и (6) этому корню отвечает следующая комбинация линейно независимых решений:
    \begin{center}
      $\gamma_1^n = P_{r-1}(n)$ \\
      \footnotesize{$P_{r-1}$ - произвольный полином в степени $r-1$}
    \end{center}

\end{document}
