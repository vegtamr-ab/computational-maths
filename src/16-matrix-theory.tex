\documentclass[a4paper,11pt]{article}

\usepackage{wrapfig}
\usepackage{amsmath}
\usepackage{pgfplots}
\usepackage[most]{tcolorbox} % для управления цветом
%Russian-specific packages
%--------------------------------------
\usepackage[T2A]{fontenc}
\usepackage[utf8]{inputenc}
\usepackage[russian, english]{babel}
%--------------------------------------

%Для выделения блока текста в рамку
\definecolor{block-gray}{gray}{0.95} % уровень прозрачности (1 - максимум)
\newtcolorbox{importantblock}{colback=block-gray,grow to right by=-10mm,grow to left by=-10mm,
boxrule=0pt,boxsep=0pt,breakable} % настройки области с изменённым фоном

\definecolor{lemonchiffon}{rgb}{1.0, 0.98, 0.8}
\newtcolorbox{mainblock}{colback=lemonchiffon,grow to right by=-10mm,grow to left by=-10mm,
boxrule=0pt,boxsep=0pt,breakable} % настройки области с изменённым фоном

\makeatletter
\newcommand{\settag}[1]{
  \tag*{(#1)\qquad}
  \edef\@currentlabel{\theequation#1}}
\makeatother

\title{16. Обратная матрица, собственные числа и векторы. Задачи на матрицы.
       Норма матрицы, сходимость матричного степенного ряда, функции от матрицы}
\author{Андрей Бареков \and Ярослав Пылаев \and По лекциям Устинова С.М.}
\date{\today}

\begin{document}
\maketitle
\newpage

\section{Элементы теории матриц}
\begin{equation*}
  A =
  \begin{pmatrix}
    a_{11} & a_{12} & \dots & a_{1n} \\
    a_{21} & a_{22} & \dots & a_{2n} \\
    \hdotsfor{4} \\
    a_{m1} & a_{m2} & \dots & a_{mn}
  \end{pmatrix}
\end{equation*}
\begin{equation*}
  \begin{vmatrix}
    a_{11} & a_{12} \\
    a_{21} & a_{22}
  \end{vmatrix}
  ,\hspace{7pt}
  det(A), \hspace{7pt}
  \underbrace
  {
    \begin{Vmatrix}
      a_{11} & a_{12} \\
      a_{21} & a_{22}
    \end{Vmatrix}
  }_{\text{норма матрицы}}
  \hspace{7pt} \text{определитель матрицы}.
\end{equation*}
\subsection{Определения}
\begin{itemize}
  \item Транспонированная матрица $B = A^T$, $b_{ij}=a_{ji}$.
  \item $X=0$ - нулевая матрица, все ее элементы равны $0$.
  \item Матрица называется левый (нижний) треугольник, если все ее элементы \textit{выше} главной диагонали равны $0$.
  \item Соответственно - правый (верхний) треугольник, если элементы \textit{ниже} главной диагонгали равны $0$.
  \item Диагональная матрица - все элементы, кроме диагональных, равны $0$:
    $D = diag(d_{11}, d_{22},\dots, d_{mn})$. \\
    Диагональная матрица называется \textit{единичной}, если все диоагональные элементы равны $1$
    \begin{equation*}
      E =
      \begin{pmatrix}
        1 & \dots & 0 \\
         & 1 &  \\
        0 & \dots & 1
      \end{pmatrix}
      .
    \end{equation*}
\end{itemize}
\subsection{Операции}
  \begin{enumerate}
    \item $C=A+B$, \hspace{7pt}
      $c_{ij}=a_{ij}+b_{ij}$.
    \item $C=\alpha A$, \hspace{7pt}
      $c_{ij}=\alpha a_{ij}$,
      $\alpha$ - скаляр.
    \item $C=AB$, \hspace{7pt}
      \[c_{ij} = \sum_{k=1}^n a_{ik}b_{kj},\] \\
      $AB \ne BA$, если же $AB=BA$, то такие матрицы называются коммутирующие или перестановочные.
    \item Вместо операции деления вводится \textit{умножение на обратную матрицу}; \\
      Обратная матрица: \[X=A^{-1},\] \\
      \[AX=E, XA=E\] \\
      Пусть $x_k$ - $k$-й столбец обратной матрицы, а $e_k$ - $k$-й столбец единичной матрицы,
      тогда $Ax_k=e_k, \hspace{5pt} k=1,2,\dots,n$. Если $det(A) \ne 0$, то система имеет единственное решение,
      если $det(A)=0$, то матрица называется особенной.
  \end{enumerate}

\section{Собственные значения и векторы}
\begin{equation}
  Ax = \lambda x
\end{equation}
\begin{center}
  $\lambda$ - \textit{собственные значения}, вектор $x$ - \textit{собственный вектор}.
\end{center}
\begin{equation}
  (A - \lambda E)x = 0
\end{equation}
Пусть $det(A-\lambda E) \ne 0$, тогда система (2) имеет единственное решение $x=0$. \\
Чтобы найти вектор $x$, отличный от нулевого, нужно потребовать, чтобы определитель
\begin{equation}
  det(A-\lambda E)=0
\end{equation}
\begin{gather*}
  det(A-\lambda E) = (-1)^n\lambda^n+b_1\lambda^{n-1}+\dots+b_{n-1}\lambda+b_n = 0 \\
  \small\text{{характеристическое уравнение, его $n$ корней - собственные значения.}}
\end{gather*}
Для каждого $\lambda_i$ можно найти решение $x_i$ однородной системы $Ax_i=\lambda_ix_i$, т.е. собственный вектор. \\

Таким образом, для матрицы малой размерности может быть предложен следующий алгоритм:
\begin{itemize}
  \item из уравнения (3) находим все собственные значения,
  \item для каждого собственного значения из системы (2) находим собственный вектор.
\end{itemize}

\section{Задачи на матрицы}
\begin{enumerate}
  \item $(AB)^T = B^T A^T$
  \item $(AB)^{-1} = B^{-1}A^{-1}$
  \item $(A^T)^{-1} = (A^{-1})^T$
  \item $DA$ - при умножении на диагональную матрицу \textit{слева} все \textit{строки} матрицы умножаются на соответствующие диагональные элементы.
  \item $AD$ - при умножении на диагональную матрицу \textit{справа} все \textit{столбцы} матрицы умножаются на соответствующие диагональные элементы.
  \item
    \[\begin{pmatrix}
      * & 0 & 0 \\
      * & * & 0 \\
      * & * & *
    \end{pmatrix} \times
    \begin{pmatrix}
      * & 0 & 0 \\
      * & * & 0 \\
      * & * & *
    \end{pmatrix} =
    \begin{pmatrix}
      * & 0 & 0 \\
      * & * & 0 \\
      * & * & *
    \end{pmatrix} \hspace{7pt} \text{получили тот же тип.}\]
  \item
    \[\begin{pmatrix}
    * & 0 & 0 \\
    * & * & 0 \\
    * & * & *
    \end{pmatrix}^{-1} =
    \begin{pmatrix}
      * & 0 & 0 \\
      * & * & 0 \\
      * & * & *
    \end{pmatrix}\]
  \item Собственные значения диагональной матрицы равны ее диагональным элементам.
  \item Собственные значения треугольной матрицы тоже равне ее диагональным элементам.
  \item Сумма собственных значений матрицы равна сумме ее диагональных элементов
    \[\sum_{k=1}^N \lambda_k = \underbrace{\sum_{k=1}^N a_{kk}}_{\text{след матрицы}},\]
    а произведение - ее определителю
    \[\prod_{k=1}^N \lambda_k = det(A).\]
    Если матрица особенная ($det(A)=0$), то хотя бы одно собтвенное значение равно $0$.
\end{enumerate}

\section{Дифференцирование и интегрирование матриц по параметру}
\begin{gather*}
  B(t) = \frac{d}{dt}A(t), \hspace{5pt} b_{ij}(t)=\frac{d}{dt}a_{ij}(t), \\
  F = \int_a^b A(t)\ dt, \hspace{5pt} f_{ij} = \int_a^b a_{ij}(t)\ dt.
\end{gather*}
\newpage
\marginpar
{
  \vspace{4mm}
  \footnotesize \[A^k = \underbrace{AA\dots A}_{k}, \hspace{2pt} A^0 \equiv E\]
}
\begin{gather*}
  \frac{d}{dt}A^2(t), \\
  B(t) = A^2(t), \hspace{5pt} b_{ij}(t) = \sum_{k=1}^n a_{ik}(t)a_{kj}(t), \\
  \frac{d}{dt}B(t) = \frac{dA}{dt}A + A\frac{dA}{dt}, \\
  \text{для } A^{17}: \frac{d}{dt}B(t) = A\frac{dA}{dt}A^{15} + A^2\frac{dA}{dt}A^{14} +\dots \\
\end{gather*}
Найдем производную от обратной матрицы
\begin{gather*}
  AA^{-1}=E, \\
  \frac{dA}{dt}A^{-1} + A\frac{d}{dt}A^{-1} = 0, \hspace{7pt}
  \frac{d}{dt}A^{-1} = - A^{-1}\frac{dA}{dt}A^{-1}.
\end{gather*}

\section{Норма матрицы $\parallel A\parallel$}
Нормой матрицы называется число, удовлетворяющее следующим условиям (аксиомам):
\begin{enumerate}
  \item норма матрицы не отрицательна $\parallel A\parallel >0$, $\parallel A\parallel =0 \Rightarrow A=0$,
  \item $\parallel \alpha A\parallel = |\alpha|\cdot\parallel A\parallel$,
  \item $\parallel A+B\parallel \hspace{3mm}\le\hspace{3mm} \parallel A\parallel + \parallel B\parallel$,
  \item $\parallel AB\parallel \hspace{3mm}\le\hspace{3mm} \parallel A\parallel \cdot \parallel B\parallel$.
\end{enumerate}
Норма называется \textit{канонической}, если дополнительно выполняются еще две аксиомы:
\begin{equation*}
  5. \parallel A\parallel \ge |a_{ij}|, \hspace{5pt} 6. \hspace{1mm}|b_{ij}>|a_{ij}| \Rightarrow \parallel B\parallel > \parallel A \parallel.
\end{equation*}
Все 3 следующие нормы являются каноническими:
\begin{gather*}
  \parallel A\parallel_I = max_i \sum_{j=1}^n |a_{ij}|, \\
  \parallel A\parallel_{II} = max_j \sum_{i=1}^n |a_{ij}|, \\
  \parallel A \parallel_{III} = \sqrt{\sum_{i=1}^n \sum_{j=1}^n |a_{ij}|^2}.\\
\end{gather*}

Примеры:

\begin{minipage}{1\linewidth}
  \begin{wraptable}{l}{5mm}
    \begin{equation*}
      A = \begin{pmatrix}
        1 & -2 & 3 \\
        -4 & 5 & -6 \\
        7 & -8 & 9
      \end{pmatrix}
    \end{equation*}
  \end{wraptable}
  \begin{align*}
    &\parallel A\parallel_I = 24, \\
    &\parallel A\parallel_{II} = 18, \\
    &\parallel A\parallel_{III} \approx 16.9.
  \end{align*}
  \begin{wraptable}{l}{5mm}
    \begin{equation*}
      A = \begin{pmatrix}
        1 & 0 & 0 \\
        0 & 1 & 0 \\
        0 & 0 & 1
      \end{pmatrix}
    \end{equation*}
  \end{wraptable}
  \begin{align*}
    &\parallel A\parallel_I = 1, \\
    &\parallel A\parallel_{II} = 1, \\
    &\parallel A\parallel_{III} = \sqrt{3}.\\
  \end{align*}
\end{minipage}
\begin{center}
  Знак сравнения между нормами $I-III$ ставить \textit{нельзя}.
\end{center}

\section{Матричный ряд}
\begin{gather*}
  P_m(A) = \alpha_0E + \alpha_1A + \alpha_2A^2 +\dots+ \alpha_mA^m = \sum_{k=0}^m \alpha_kA^k, \\
  \small{\text{$\alpha$ - скалярный коэффициент, как и прочие строчные греческие буквы},} \\
  P(A) = \sum_{k=0}^{\infty} \alpha_kA^k \hspace{5pt} \text{- матричный степенной ряд},
\end{gather*}
Матричный ряд будет сходящимся, если сходятся все $n^2$ скалярных рядов для элементов матрицы $P(A)$.
\begin{gather*}
  P(A) = \sum_{k=0}^{\infty}\alpha_kA^k, \hspace{3mm} p_{ij}, \\
  \text{Введем матрицу} \hspace{3mm} U^{(k)} = \alpha_kA^k, \hspace{3mm} u_{ij}^{(k)},
\end{gather*}
\begin{align*}
  |p_{ij}| &= \bigg|\sum_{k=0}^{\infty}u_{ij}^{(k)}\bigg| \le \sum_{k=0}^{\infty}\bigg|u_{ij}^{(k)}\bigg| \stackrel{5}{\le}
  \sum_{k=0}^{\infty}\parallel U^{(k)}\parallel = \sum_{k=0}^{\infty}\parallel \alpha_kA^k\parallel \stackrel{2}{=} \\
  &\stackrel{2}{=} \sum_{k=0}^{\infty}|\alpha_k|\parallel A^k\parallel \hspace{1mm} \stackrel{4}{\le}
  \underbrace{\sum_{k=0}^{\infty}|\alpha_k|\parallel A\parallel^k}_{\text{скалярный степенной ряд}}.
\end{align*}
В результате \textit{достаточным} условием сходимости матричного ряда является выполнение условия
\begin{equation*}
  \boxed{\parallel A\parallel < R}
\end{equation*}
являющегося, в свою очередь, условием абсолютной сходимости скалярного степенного ряда.
$R$ - радиус сходимости скалярного степенного ряда. \\

Пример:
\begin{align*}
  A = \begin{pmatrix}
    1 & -2 & 3 \\
    -4 & 5 & -6 \\
    7 & -8 & 9
  \end{pmatrix}
  &&
  \parallel A\parallel_I = 24, \parallel A\parallel_{II} = 18, \parallel A\parallel_{III} = 16.9,
\end{align*}
$R = 17.5 \hspace{5mm} \text{- ряд сходится}$, \\
$R = 16.7 \hspace{5mm} \text{- достаточное условие не выполянется, сходится ли ряд - не знаем}$. \\

Если матричный ряд сходится, то его сумму будем называть \textit{матричной функцией}. \\

Примеры матричных функций:
\begin{align*}
  &e^x=\sum_{k=0}^{\infty}\frac{x^k}{k!} \Rightarrow e^A=\sum_{k=0}^{\infty}\frac{A^k}{k!}, & R=\infty, \\
  &cos(x)=\sum_{k=0}^{\infty}\frac{(-1)^kx^{2k}}{(2k)!} \Rightarrow cos(A)=\sum_{k=0}^{\infty}\frac{(-1)^kA^{2k}}{(2k)!},
  & R=\infty, \\
  &\underbrace{\frac{1}{1-x}=\sum_{k=0}^{\infty}x^k}_{\text{геом. прогрессия}} \Rightarrow (E-A)^{-1}=\sum_{k=0}^{\infty}A^k, &R= \hspace{2mm}1.
\end{align*}

\end{document}
