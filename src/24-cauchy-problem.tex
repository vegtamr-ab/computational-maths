\documentclass[a4paper,11pt]{article}

\usepackage{wrapfig}
\usepackage{amsmath}
\usepackage{pgfplots}
\usepackage{xcolor}
\usepackage[most]{tcolorbox}
%Russian-specific packages
%--------------------------------------
\usepackage[T2A]{fontenc}
\usepackage[utf8]{inputenc}
\usepackage[russian, english]{babel}
%--------------------------------------

\definecolor{lemonchiffon}{rgb}{1.0, 0.98, 0.8}
\newtcolorbox{mainblock}{colback=lemonchiffon,grow to right by=-10mm,grow to left by=-10mm,
boxrule=0pt,boxsep=0pt,breakable} % настройки области с изменённым фоном
\definecolor{block-gray}{gray}{0.95} % уровень прозрачности (1 - максимум)
\newtcolorbox{importantblock}{colback=block-gray,grow to right by=-10mm,grow to left by=-10mm,
boxrule=0pt,boxsep=0pt,breakable} % настройки области с изменённым фоном

\makeatletter
\newcommand{\settag}[1]{
  \tag*{(#1)\qquad}
  \edef\@currentlabel{\theequation#1}}
\makeatother

\title{24. Задача Коши решения обыкновенных дифференциальных уравнений. Явный и неявный методы ломаных Эйлера, метод трапеций}
\author{Андрей Бареков \and Ярослав Пылаев \and По лекциям Устинова С.М.}
\date{\today}

\begin{document}
\maketitle
\newpage

\noindent Интегрируемых в явном виде дифференциальных уравений чрезвычайно мало. Поэтому столь важны численные методы. \\

\begin{equation}
  \frac{dx}{dt} = f(t, x),
  \label{eq:LDLE}
\end{equation}
\begin{gather*}
  t - \text{независимая переменная}, \, x - \text{вектор искомых функций}, \\
  x(t_0) = x_0 - \text{начальная точка Задачи Коши}.
\end{gather*}

\noindent Первоначально рассмотрим одно уравнение, хотя все полученные методы сохраняют свой внешний вид и для случая, когда $x$ - вектор.
Численное решение характеризируется \textit{устойчивостью} (характер изменения решения при внесении в него изменений) и
\textit{точностью} (отличие приближенного решения от точного). \\

\begin{importantblock}
  Задача Коши или задача с начальными условиями:
  \begin{center}
    \begin{tikzpicture}
      \begin{axis} [
        xlabel = {t}, ylabel = {x(t)},
        xmin = 0, ymin = 0,
        xtick = {}, ytick = {},
        xticklabels = {0, $t_0$, $t_1$, $t_2$, $t_3$, $t_4$, $t_5$},
        no markers
      ]

        \addplot+[smooth] [
          color = red,
          thick,
          domain = 0:7,
          samples = 300
        ] coordinates { (0,1.6)(0.25,2)(0.5,2.5)(0.75,3)(1,3.4)(1.25,3.8)(1.5,4.05)(1.75,4.3)(2,4.5)(2.25,4.6)(2.5,4.7)(2.75,4.8)
                        (3,4.7)(3.25,4.5)(3.5,4.4)(3.75,4.2)(4,4.1)(4.25,4.1)(4.5,4.2)(4.75,4.35)(5,4.5)(5.25,4.65)(5.5,4.95)(5.75,5.05)(6,5.1) };

        \addplot+[const plot] [
          color = blue,
          thin,
          style = dashed,
        ] coordinates { (1,0)(1,3.4) };

        \addplot+[const plot] [
          color = blue,
          thin,
          style = dashed,
        ] coordinates { (2,0)(2,4.5) };

        \addplot+[const plot] [
          color = blue,
          thin,
          style = dashed,
        ] coordinates { (3,0)(3,4.7) };

        \addplot+[const plot] [
          color = blue,
          thin,
          style = dashed,
        ] coordinates { (4,0)(4,4.1) };

        \addplot+[const plot] [
          color = blue,
          thin,
          style = dashed,
        ] coordinates { (5,0)(5,4.5) };
      \end{axis}
    \end{tikzpicture}
  \end{center}
  Исходное дифференциальное уравнение сводится к некоторому разностному, которое потом решается пошаговым методом.
  \[t_n = t_0 + n \times h^[\footnote{шаг интегрирования или шаг дискретности}^]\]
  \[x_n \stackrel{\mathrm{def}}{=} x(n),\quad f_n \stackrel{\mathrm{def}}{=} f(t_n, x_n)\]
\end{importantblock}
Проинтегрируем уравнение (\ref{eq:LDLE}) на промежутке $[t_n, t_{n+1}]$:
\begin{equation}
  x_{n+1} = x_n + \int_{t_n}^{t_{n+1}} f(\tau, x(\tau))\,d\tau
  \label{eq:Solution}
\end{equation}
Различные методы отличаются друг от друга способом вычисления интеграла в формуле (\ref{eq:Solution}).
Применяем формулу \textit{левых прямоугольников}:
\begin{gather*}
  \int_a^b G(x)\,dx \approx (b-a)G(a) \\
  \boxed{x_{n+1} = x_n + hf(t_n, x_n)}
\end{gather*}
\begin{center}
  \textbf{\small{Явный метод ломаных Эйлера}}
\end{center}
Применяем формулу \textit{правых прямоугольников}:
\begin{gather*}
  \int_a^b G(x)\,dx \approx (b-a)G(b) \\
  \boxed{x_{n+1} = x_n + hf(t_{n+1}, x_{n+1})}
\end{gather*}
\begin{center}
  \textbf{\small{Неявный метод ломаных Эйлера}}
\end{center}
Применяем формулу \textit{трапеций}:
\begin{gather*}
  \int_a^b G(x)\,dx \approx \frac{b-a}{2}(G(a) + G(b)) \\
  \boxed{x_{n+1} = x_n + \frac{h}{2}(f(t_n, x_n) + f(t_{n+1}, x_{n+1}))}
\end{gather*}
\begin{center}
  \textbf{\small{Неявный метод трапеций}}
\end{center}
В неявных методах на каждом шаге приходится решать нелинейное уравнение относительно $x_{n+1}$. Дальнейшее же использование
квадратичных формул непосредственно затруднено, т.к. они трубуют знания $x(t)$ внутри промежутка $t_n,\, t_{n+1}$.

\end{document}
