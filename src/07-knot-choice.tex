\documentclass[a4paper,11pt]{article}

\usepackage{wrapfig}
\usepackage{amsmath}
\usepackage{pgfplots}
\usepackage[most]{tcolorbox} % для управления цветом
%Russian-specific packages
%--------------------------------------
\usepackage[T2A]{fontenc}
\usepackage[utf8]{inputenc}
\usepackage[russian, english]{babel}
%--------------------------------------

%Для выделения блока текста в рамку
\definecolor{block-gray}{gray}{0.95} % уровень прозрачности (1 - максимум)
\newtcolorbox{importantblock}{colback=block-gray,grow to right by=-10mm,grow to left by=-10mm,
boxrule=0pt,boxsep=0pt,breakable} % настройки области с изменённым фоном

\title{07. Выбор узлов интерполирования. Интерполяционный полином Ньютона для равно- и неравноотстоящих узлов.}
\author{Андрей Бареков \and Ярослав Пылаев \and По лекциям Устинова С.М.}
\date{\today}

\begin{document}
\maketitle
\newpage

\section{Выбор узлов интерполирования}
Реально повлиять на величину погрешности можно только минимизируя величину $|\omega(x)|$, что делается выбором узлов интерполирования.
\[|\omega(x)| = |(x-x_0)(x-x_1)\dots(x-x_m)| \rightarrow \min\]

  \subsection{Случай 1}
  \begin{itemize}
    \item Задана степень полинома $m$
    \item Есть таблица большой длины ($>(m + 1)$)
    \item \textit{Задана} точка $x^*$, в которой оценивается погрешность
  \end{itemize}
  Очевидно, что лучший выбор - узлы, ближайшие к $x^*$.

  \subsection{Случай 2}
  \begin{itemize}
    \item Задан промежуток интерполирования $[a, b]$
    \item Задана степень полинома $m$
    \item Точка $x^*$ заранее \textit{неизвестна}
  \end{itemize}
  Требуется выбрать узлы так, чтобы в худшем случае погрешность была бы минимальной.
  \[\max |\omega(x)| \rightarrow \min, \, x \in [a, b]\]
  Интуитивно ясно, что узлы следует располагать симметрично относительно середины промежутка. \\
  Рассмотрим случай равноотстоящих узлов: \\
  \begin{minipage}{1\linewidth}
    \begin{wrapfigure}{l}{8cm}
    \begin{tikzpicture}
      \begin{axis} [
        xlabel = {}, ylabel = {},
        xmin = 0, ymin = 0,
        xtick = {}, ytick = {},
        no markers
      ]
        \addplot+[smooth] [
        color = red,
        thick,
        domain = 0:6,
        samples = 300
      ] coordinates { (0,0)(0.17,3)(0.34,8)(0.5,9)(0.67,8)(0.84,3)(1,0) };

        \addplot+[smooth] [
        color = red,
        ultra thin,
        domain = 0:6,
        samples = 300
      ] coordinates { (1,0)(1.17,2)(1.34,5.3)(1.5,6)(1.67,5.3)(1.84,2)(2,0) };

        \addplot+[smooth] [
        color = red,
        ultra thin,
        domain = 0:6,
        samples = 300
      ] coordinates { (2,0)(2.17,1)(2.34,2.6)(2.5,3)(2.67,2.6)(2.84,1)(3,0) };

        \addplot+[smooth] [
        color = red,
        thick,
        domain = 0:6,
        samples = 300
      ] coordinates { (3,0)(3.17,2)(3.34,5.3)(3.5,6)(3.67,5.3)(3.84,2)(4,0) };

        \addplot+[smooth] [
        color = red,
        ultra thin,
        domain = 0:6,
        samples = 300
      ] coordinates { (4,0)(4.17,3)(4.34,8)(4.5,9)(4.67,8)(4.84,3)(5,0) };

      \end{axis}
    \end{tikzpicture}
    \end{wrapfigure}
    Для уменьшения погрешности узлы интерполирования необходимо сместить ближе к краям промежутка, чтобы "колокольчики" стали примерно одинаковой высоты. \\
    Оптимальный выбор узлов отвечает нулям ортогональных полиномов Чебышёва.
  \end{minipage}

\newpage
  \subsection{Как на практике оценивается погрешность}
  Непосредственно оценивают погрешность по формуле выше редко, т.к. трудно оценивать производную. В инженерной практике часто используется следующий пример:
  \begin{enumerate}
    \item Строим $Q_m(x)$
    \item Добавляем $x_{m+1}$, ещё один узел
    \item Строим $Q_{m+1}(x)$
    \item Оцениваем погрешность по разности значений этих полиномов в нужной точке
  \end{enumerate}
  Использовать для этих целей полином Лагранжа неэффективно. На практике хотелось бы построить интерполяционный полином так, чтобы полином степени $m+1$
  получался добавлением к нему какого-то слагаемого степени $m$, как это происходило в построении степенного ряда Тейлора.

\section{Полином Ньютона}
Рассмотрим первую разделённую разность:
\[f(x; x_0) = \frac{f(x) - f(x_0)}{x - x_0}\]
\begin{equation}
  f(x) = \underbrace{f(x_0)}_{Q_0} + \underbrace{(x - x_0)(f(x; x_0))}_{R_0}
  \label{eq:DDNewt}
\end{equation}
Рассмотрим вторную разделенную разность:
\[f(x; x_0; x_1) = \frac{f(x; x_0) - f(x_0; x_1)}{x - x_1}\]
Выразим первую разделённую разность через вторую и подставим в формулу (\ref{eq:DDNewt}):
\[f(x) = \underbrace{\underbrace{f(x_0)}_{Q_0} + (x - x_0)f(x_0; x_1)}_{Q_1} + \underbrace{(x - x_0)(x - x_1)f(x; x_0; x_1)}_{R_1}\]
Продолжая этот процесс, и выражая вторую разделённую разность через третью, третью через четвертую и т.д., получаем:
\[f(x) = Q_m(x) + \underbrace{(x - x_0)(x - x_1)\dots(x - x_m)f(x; x_0; x_1; \dots; x_m)}_{R_m}\]
\vspace{10mm}
\begin{equation}
  \begin{aligned}
    Q_m(x) &= f(x_0) + (x - x_0)f(x_0; x_1)+(x - x_0)(x - x_1)f(x_0; x_1; x_2) +\\
          & + \dots + (x - x_0)\dots(x - x_{m-1})f(x_0; x_1; \dots; x_m) \\
  \end{aligned}
  \label{eq:GenNewtPol}
\end{equation}
\begin{center}
  \textbf{Полином Ньютона для неравноотстоящих узлов}
\end{center}

Если узлы интерполирования равноотстоящие, то в формуле (\ref{eq:GenNewtPol}) можно заменить разделённые разности на конечные. С этой целью выполним замену переменных:
\begin{equation*}
  \begin{cases}
    x = x_0 + ht \\
    x_k = x_0 + kh
  \end{cases}, \, \text{в узлах $t$ - целое}
\end{equation*}

\begin{gather}
  f(x_0; \dots; x_m) = \frac{\Delta^mf(x_0)}{m!\ h^m}
  \label{eq:ConclusionFormls}
\end{gather}
\begin{equation*}
  (x - x_k) = h(t - k)
\end{equation*}

\begin{flushright}
  \footnotesize {
    Знаем, что $x = x_0 + ht$ \\
    $x-x_1 = x-x_0-h = th-h = h(t-1)$ \\
    Тогда $x-x_k = h(t-k)$ \\
    $\Rightarrow (x-x_0)(x-x_1) = t(t-1)h^2$ }
\end{flushright}

Тогда формула (\ref{eq:GenNewtPol}) при подставновке (\ref{eq:ConclusionFormls}) принимает следующий вид:
\begin{align*}
  Q_m(x_0 + ht) & = f(x_0) + \frac{t}{1!}\Delta f(x_0) + \frac{t(t-1)}{2!}\Delta^2f(x_0) + \\
                & + \dots + \frac{t(t-1)\dots(t-m+1)}{m!}\Delta^mf(x_0)
\end{align*}
  \begin{center}
  \textbf{Полином Ньютона для равноотстоящих узлов}
\end{center}

\vspace{5mm}
Полиномы строятся последовательно в соответствии со следующей таблицей:

\begin{minipage}{\linewidth}
  \begin{wraptable}{l}{6cm}
    \begin{tabular}{ c|c|c|c }
      $x$ & $f(x)$ & $\Delta f(x)$ & $\Delta^2f(x)$ \\
      \hline
      $x_0$ & $f(x_0)$ & $\underline{\Delta f(x)}$ & $\underline{\underline{\Delta^2f(x)}}$ \\
      $x_1$ & $\underline{f(x_1)}$ & $\underline{\underline{\Delta f(x_1)}}$ & \\
      $x_2$ & $\underline{\underline{f(x_2)}}$ & &
    \end{tabular}
  \end{wraptable}
    Каждая новая степень полинома требует построения очередной диагонали в таблице.
\end{minipage}

\end{document}
