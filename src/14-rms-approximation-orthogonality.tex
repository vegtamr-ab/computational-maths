\documentclass[a4paper,11pt]{article}

\usepackage{wrapfig}
\usepackage{amsmath}
\usepackage{pgfplots}
\usepackage[most]{tcolorbox} % для управления цветом
%Russian-specific packages
%--------------------------------------
\usepackage[T2A]{fontenc}
\usepackage[utf8]{inputenc}
\usepackage[russian, english]{babel}
%--------------------------------------

%Для выделения блока текста в рамку
\definecolor{block-gray}{gray}{0.95} % уровень прозрачности (1 - максимум)
\newtcolorbox{importantblock}{colback=block-gray,grow to right by=-10mm,grow to left by=-10mm,
boxrule=0pt,boxsep=0pt,breakable} % настройки области с изменённым фоном

\makeatletter
\newcommand{\settag}[1]{
  \tag*{(#1)\qquad}
  \edef\@currentlabel{\theequation#1}}
\makeatother

\title{14. Среднеквадратичная аппроксимация (непрерывный случай). Понятие ортогональности}
\author{Андрей Бареков \and Ярослав Пылаев \and По лекциям Устинова С.М.}
\date{\today}

\begin{document}
\maketitle
\newpage

Продолжение вопроса "13. Среднеквадратичная аппроксимация..."
\section{Непрерывный случай}
Функция задана непрервным образом на промежутке $[a,b]$.
В среднеквадратичном критерии вместо суммы возникает определенный на промежутке интеграл
\begin{gather*}
  \rho^2 = \int_a^b p(x)\bigg( Q_m(x)-f(x) \bigg)^2 dx \rightarrow min
\end{gather*}
\begin{flushright}
  \small
  $x\in [a,b]$, $p(x) - \text{весовая функция}$, \\
  $Q_m(x) = \sum_{k=0}^m a_k\varphi_k(x)$
\end{flushright}
Необходимое услвоие экстремума: $\frac{\partial \rho^2}{\partial a_k} = 0$, $k=0,1,2,\dots,m$. \\
Система уравнений выглядит следующим образом: \\
\begin{equation}
  \begin{split}
    a_0\int_a^b p(x)\varphi_0(x)\varphi_k(x)dx +\dots+ a_m\int_a^b p(x)\varphi_m(x)\varphi_k(x)dx = \\
    = \int_a^b p(x)f(x)\varphi_k(x)dx
  \end{split}
  \label{eq:RmsEq}
\end{equation}
Решаем систему (\ref{eq:RmsEq}) и находим коэффициенты $a_k$. \\

\section{Понятие ортогональности}
Система резко упрощается, если функции ${\varphi_k(x)}$ ортогональные.
Последовательность функций ${\varphi_k(x)}$ называется \textit{ортогональной} на промежутке $[a,b]$ с весом $p(x)$, если
\begin{equation}
 \bigg(\varphi_k(x), \varphi_i(x)\bigg) = \int_a^b p(x)\varphi_k(x)\varphi_i(x)dx =
 \begin{cases}
   0, & i \ne k, \\
   A>0, & i=k
 \end{cases}
\end{equation}

Если $A=1$, то функции называются \textit{ортонормированными}. В этом случае все, кроме одного, интегралы в левой части
(\ref{eq:RmsEq}) обращаются в $0$, кроме одного, и для $a_k$ получается готовое выражение:
\begin{equation*}
  a_k = \frac{\int_a^b p(x)f(x)\varphi_k(x)dx}{\int_a^b p(x)\varphi_k^2(x)dx}
\end{equation*}
Если функции еще и \textit{нормированные}, то знаменатель равен $1$.

\end{document}
