\documentclass[a4paper,11pt]{article}

\usepackage{wrapfig}
\usepackage{amsmath}
\usepackage{pgfplots}
\usepackage{xcolor}
\usepackage[most]{tcolorbox}
%Russian-specific packages
%--------------------------------------
\usepackage[T2A]{fontenc}
\usepackage[utf8]{inputenc}
\usepackage[russian, english]{babel}
%--------------------------------------

\definecolor{lemonchiffon}{rgb}{1.0, 0.98, 0.8}
\newtcolorbox{mainblock}{colback=lemonchiffon,grow to right by=-10mm,grow to left by=-10mm,
boxrule=0pt,boxsep=0pt,breakable} % настройки области с изменённым фоном
\definecolor{block-gray}{gray}{0.95} % уровень прозрачности (1 - максимум)
\newtcolorbox{importantblock}{colback=block-gray,grow to right by=-10mm,grow to left by=-10mm,
boxrule=0pt,boxsep=0pt,breakable} % настройки области с изменённым фоном

\makeatletter
\newcommand{\settag}[1]{
  \tag*{(#1)\qquad}
  \edef\@currentlabel{\theequation#1}}
\makeatother

\title{28. Метод Ньютона в неявных алгоритмах решения дифференциальных уравнений}
\author{Андрей Бареков \and Ярослав Пылаев \and По лекциям Устинова С.М.}
\date{\today}

\begin{document}
\maketitle
\newpage

Методы, пригодные для решения жёстких систем, как правило, являются неявными. Для нахождения $x_{n+1}$ на каждом шаге решаются нелинейные уравнения методом Ньютона.
  В качестве примера рассмотрим неявный метод ломаных Эйлера:
\begin{equation}
  \frac{dx}{dt} = f(t, x)  
\end{equation}
\[x_{n+1} = x_n + hf(t_{n+1}, x_{n+1})\]
\[F(z) = z - x_n - hf(t_{n+1}, z) = 0\]
\begin{equation}
  \begin{cases}
    \frac{\partial F}{\partial z}z_k \varepsilon_k = -F(z_k) \\
    z_{k+1} = z_k + \varepsilon_k
  \end{cases}
  \label{eq:NM}
\end{equation}
Решение системы (\ref{eq:NM}) это $x_{n+1}$. Применяем к уравнению (\ref{eq:NM}) метод Ньютона:
\[\frac{\partial F}{\partial z} = E - h\frac{\partial f(t_{n+1}, z)}{\partial z}\]
$z_k$ - это $k$-тое приближение к $x_{n+1}$. \\
В качестве начального приближения $z_0$ можно выбрать решение на предыдущем шаге $x_n$. \\
Второй способ - сделать один шаг явного метода ломаных Эйлера: $z_0 = x_n + hf(t_n, x_n)$. \\
\begin{importantblock}
  Большую популярность получил модифицированный метод Ньютона. $\frac{dF}{dz}(x_0)$ вычисляется \underline{однократно} в начальной точке. Однократно
    выполняется её $LU$-разложение программой $DECOMP$, и на последующих шагах в методе Ньютона используется только $SOLVE$. Матрица вычисляется заново
    только когда метод Ньютона перестаёт сходиться за три итерации.
\end{importantblock}
Увеличение объёма работы на одном шаге в неявных методах по сравнению с явными полностью окупается выигрышем в величине шага для жёстких систем.

\end{document}
