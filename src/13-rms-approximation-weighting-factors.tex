\documentclass[a4paper,11pt]{article}

\usepackage{wrapfig}
\usepackage{amsmath}
\usepackage{pgfplots}
\usepackage[most]{tcolorbox} % для управления цветом
%Russian-specific packages
%--------------------------------------
\usepackage[T2A]{fontenc}
\usepackage[utf8]{inputenc}
\usepackage[russian, english]{babel}
%--------------------------------------

%Для выделения блока текста в рамку
\definecolor{block-gray}{gray}{0.95} % уровень прозрачности (1 - максимум)
\newtcolorbox{importantblock}{colback=block-gray,grow to right by=-10mm,grow to left by=-10mm,
boxrule=0pt,boxsep=0pt,breakable} % настройки области с изменённым фоном

\makeatletter
\newcommand{\settag}[1]{
  \tag*{(#1)\qquad}
  \edef\@currentlabel{\theequation#1}}
\makeatother

\title{13. Среднеквадратичная аппроксимация (дискретный случай). Понятие веса}
\author{Андрей Бареков \and Ярослав Пылаев \and По лекциям Устинова С.М.}
\date{\today}

\begin{document}
\maketitle
\newpage

\section{Постановка задачи}
\begin{center}
  \begin{tikzpicture}
    \begin{axis} [
      xlabel = {}, ylabel = {},
      xmin = 0, ymin = 0,
      xtick = {}, ytick = {},
      no markers
      ]
      \addplot [
      color = blue,
      thick,
      domain = 0:4,
      samples = 300,
      only marks
      ] coordinates { (0,0)(0.25,1)(0.5,2.5)(0.6,2.5)(1,3.8)(1.1,3.7)(1.2,4.4)(1.4,4.3)(1.45,4.6)(1.6,4.5)(1.8,4.9)(2,7) };
      \addplot [
      color = red,
      ultra thick,
      domain = 0:4,
      samples = 300,
      ] coordinates { (0,0)(0.25,1)(0.5,2.5)(0.6,2.5)(1,3.8)(1.1,3.7)(1.2,4.4)(1.4,4.3)(1.45,4.6)(1.6,4.5)(1.8,4.9) };
      \addplot [
      color = green,
      ultra thick,
      domain = 0:4,
      samples = 300,
      smooth
      ] coordinates { (0,0)(0.25,1.1)(0.5,2.1)(1,3.7)(1.2,4.2)(1.4,4.5)(1.6,4.7)(1.8,4.9) };

      \addlegendentry{data}
      \addlegendentry{$f_1(x)$}
      \addlegendentry{$f_2(x)$}
    \end{axis}

  \end{tikzpicture}
\end{center}

Маловероятно, что кривая $f_1(x)$, получившаяся в результате \\ интерполирования по эксперементальных данным (точки на рисунке),
имеет вид исходной зависимости. Скорее вид кривой $f_2(x)$ отвечает этой зависимости, а отклонение экспериментальных данных
обусловлено сравнительно большой погрешностью измерений. В таком случае или, если аппроксимируемая функция задана
на промежутке $[a, b]$ непрерывно, используют $\textit{среднеквадратичный критерий}$.\\

\textit{Задача} среднеквадратичной аппроксимации (или "метода наименьших квадратов"): так подобрать аппроксимирующую функцию $Q(x)$
\begin{itemize}
  \item чтобы квадрат расстояния между $Q(x)$ и $f(x)$ был минимальным
    \begin{equation*}
      \rho^2 = \parallel Q(x)-f(x) \parallel ^2 = (Q(x)-f(x), Q(x)-f(x)) \rightarrow min
    \end{equation*}
    \begin{center}
      \small{Линейное пространство со скалярным произведением}
    \end{center}
  \item $Q(x)$ выбирается в виде обобщенного многочлена:
    \begin{equation}
      Q_m(x) = a_0\varphi_0(x) + a_1\varphi_1(x) +\dots+ a_m\varphi_m(x) = \sum_{k=0}^{m}a_k\varphi_k(x)
    \end{equation}
    \begin{center}
      \small{{$\varphi_k$} - заданный набор линейно независимых функций, \\
      коэффициенты $a_k$ подлежат определению}
    \end{center}
\end{itemize}

\section{Дискретный случай}
Коэффициенты $a_k$ выбираются из условия минимума $\rho^2$
\begin{equation}
  \rho^2 = (Q(x)-f(x), Q(x)-f(x)) = \sum_{i=1}^{N}(Q_m(x_i)-f(x_i))^2 \rightarrow min.
  \label{eq:RMScond}
\end{equation}
Три варианта соотношения чисел $N$ и $m$:
\begin{enumerate}
  \item $N=m+1$. Число коэффициентов $a_k$ равно числу точек таблицы, через которые проходит интерполяционный полином,
    являющийся \textit{единственным решением}; $\rho_{min}^2=0$.
  \item $N<m+1$. Задача имеет \textit{бесконечное множество решений}, при этом $\rho_{min}^2=0$.
  \item $N>m+1$. Задачав имеет \textit{единственное решение}, но $\rho_{min}^2 \ne 0$. \\
    Это типичный случай среднеквадратичной аппроксимации, \\ на практике $N \gg m+1$.
\end{enumerate}
Необходимое условие экстремума:
\begin{align*}
  \frac{\partial \rho^2}{\partial a_k} & = 0, & k=0,1,2,\dots,m.
\end{align*}
\begin{center}
  \small{Система из $m+1$ уравнения с $m+1$ неизвестными}
\end{center}
Дифференцируем (\ref{eq:RMScond}):
\begin{gather*}
  0 = \frac{\partial \rho^2}{\partial a_k} = 2 \sum_{i=1}^{N}(Q_m(x_i)-f(x_i))\varphi_k(x_i) \\
  \small{\sum_{i=1}^NQ_m(x_i)\varphi_k({x_i}) = \sum_{i=1}^Nf(x_i)\varphi_k(x_i),}
  \hspace{10pt}
  \small{\text{при этом } Q_m(x_i) = \sum_{j=0}^m a_j\varphi_j(x_i)}
\end{gather*}
\begin{importantblock}
  \begin{gather*}
    \Rightarrow
    a_0\sum_{i=1}^N\varphi_0(x_i)\varphi_k(x_i) + a_1\sum_{i=1}^N\varphi_1(x_i)\varphi_k(x_i)
    +\dots+ \\ +a_m\sum_{i=1}^N\varphi_m(x_i)\varphi_k(x_i) = \sum_{i=1}^N f(x_i)\varphi_k(x_i) \\
    k = 0,1,\dots,m.
  \end{gather*}
\end{importantblock}
Решаем систему уравнений и находим коэффициенты $a_k$.

\newpage
\section{Понятие веса}
На практике возникает ситуация, когда различным экспериментальным точкам доверяют в разной степени,
это можно учесть введением весовых коэффициентов
\begin{equation}
  \rho^2 = \sum_{i=1}^N \underbrace{p_i}_{p_i\footnotesize{>0}}\bigg(Q_m(x_i)-f(x_i)\bigg)^2 \rightarrow min.
\end{equation}

Для тех точек, которым мы доверяем больше и к которым \\ аппроксимирующую функцию хотим провести ближе,
весовые коэффициенты нужно выбирать больше. \\
Обращаясь к практике, положительные весовые коэффициенты $p_i$ часто задают так, чтобы их сумма была равна, например, $1$ или $100$.
Важным является отношение $p_i$ друг к другу.

\end{document}
