\documentclass[a4paper,11pt]{article}

\usepackage{wrapfig}
\usepackage{amsmath}
\usepackage{pgfplots}
\usepackage{xcolor}
\usepackage[most]{tcolorbox}
%Russian-specific packages
%--------------------------------------
\usepackage[T2A]{fontenc}
\usepackage[utf8]{inputenc}
\usepackage[russian, english]{babel}
%--------------------------------------

\definecolor{lemonchiffon}{rgb}{1.0, 0.98, 0.8}
\newtcolorbox{mainblock}{colback=lemonchiffon,grow to right by=-10mm,grow to left by=-10mm,
boxrule=0pt,boxsep=0pt,breakable} % настройки области с изменённым фоном
\definecolor{block-gray}{gray}{0.95} % уровень прозрачности (1 - максимум)
\newtcolorbox{importantblock}{colback=block-gray,grow to right by=-10mm,grow to left by=-10mm,
boxrule=0pt,boxsep=0pt,breakable} % настройки области с изменённым фоном

\makeatletter
\newcommand{\settag}[1]{
  \tag*{(#1)\qquad}
  \edef\@currentlabel{\theequation#1}}
\makeatother

\title{10. Общий подход к построению квадратурных формул. Квадратурные формулы Ньютона-Котеса, Чебышёва, Гаусса}
\author{Андрей Бареков \and Ярослав Пылаев \and По лекциям Устинова С.М.}
\date{\today}

\begin{document}
\maketitle
\newpage

\section{Общий подход}
Все ранее полученные (не составные) квадратурные формулы имели следующий вид:
\begin{equation}
  \int_a^b f(x)\ dx \approx \sum_{k=1}^S A_k f(x_k)
  \label{eq:QRBA}
\end{equation}
\begin{equation*}
  \underbrace{A_k \text{(веса)} \rightarrow\ S, \hspace{7pt} x_k \text{(узлы)}\rightarrow S}_{2S}
\end{equation*}
Выдвигается идея, выбрать $A_k$ и $x_k$ так, чтобы формула (\ref{eq:QRBA}) была точна для полиномов заданной степени. \\
Если $f(x)$ хорошо описывается полиномом этой степени, то и ответ будет хороший. В противном случае, применяем составные формулы. \\

Потребуем, чтобы формула (\ref{eq:QRBA}) была точна для полинома $N$ степени.
\begin{align*}
  1&. \hspace{7pt}f(x) = \alpha - const, \\
  2&. \hspace{7pt}f(x) = x, \\
  3&. \hspace{7pt}f(x) = x^2, \\
  \dots \\
  N+1&. \hspace{7pt}f(x) = x^N
\end{align*}
\begin{importantblock}
  \begin{equation}
    \begin{split}
      &\sum_{k=1}^S A_k = b - a \\
      &\sum_{k=1}^S A_k x_k = \frac{b^2-a^2}{2} \\
      &\sum_{k=1}^S A_k x_k^2 = \frac{b^3-a^3}{3} \\
      &\cdots \cdots \cdots \\
      &\sum_{k=1}^S A_k x_k^N = \frac{b^{N+1}-a^{N+1}}{N+1}.
    \end{split}
    \label{eq:QRSys}
  \end{equation}
\end{importantblock}
Различные семейства квадратурных формул отличаются друг от друга выбором $\underbrace{x_k}_{\text{узлов}}$ и $\underbrace{A_k}_{\text{весов}}$, на которые
  накладываются дополнительные условия.

\section{Семейство квадратурных формул Ньютона-Котеса}
\begin{importantblock}
  Узлы $x_k$ - равноотстоящие.
\end{importantblock}
\begin{align*}
  h = \frac{b-a}{S-1}, && x_k=a+(k-1)h, && x_1=a, && x_S=b.
\end{align*}
Для равноотстоящих узлов при использовании составных квадратурных формул ранее вычисленные $f(x_k)$ не надо вычислять вновь, что
уменьшает объём вычислений примерно в два раза. \\
Система (\ref{eq:QRSys}) имеет при этом единственное решение.
\begin{importantblock}
  Формулы Ньютона-Котеса гарантированно точны для полиномов степени $S-1$.
\end{importantblock}
Если решать систему (\ref{eq:QRSys}) для $S = 1, 2, 3\dots$, то получим квадратурные формулы, которые получали из интерполяционных полиномов.

\section{Семейство квадратурных формул Чебышёва}
\begin{importantblock}
  Формулы Чебышёва наиболее часто используются, когда $f(x_k)$ измерены с заметной погрешностью. С этой целью $A_k$ выбираются одинаково.
\end{importantblock}
Из уравнения 1 системы (2) видим, что $A = \frac{b-a}{S}$, тогда:
\[\int_a^b f(x)\ dx \approx \frac{b-a}{S} \sum_{k=1}^S f(x_k)\]
Решение системы (2) должно существовать, быть единственным, и все $x_k$ должны принадлежать промежутку $(a, b)$. \\
Дальнейшее повышение точности реализуется за счет составных квадратурных формул.
\begin{importantblock}
  Формулы Чебышёва гарантированно точны для полиномов степени $S$, при этом они существуют только для $S = 1-7, 9$.
\end{importantblock}

\section{Семейство квадратурных формул Гаусса}
\textit{Формулы наивысшей алгебраической степени точности.}
\begin{importantblock}
  В формулах Гаусса нет никаких дополнительных ограничений, и все $2S$ параметров направлены на решение системы (\ref{eq:QRSys}).
\end{importantblock}
Система (\ref{eq:QRSys}) имеет единственное решение для любого значения $S$ для формул Гаусса.
\begin{importantblock}
  Формулы Гаусса гарантированно точны для полиномов степени $2S-1$
\end{importantblock}

\begin{mainblock}
  Значения $A_k$ и $x_k$ для всех квадратурных формул заносят в справочники для стандартного промежутка (чаще всего это $[-1; 1]$ или $[0; 1]$), а для
    произвольного промежутка $[a; b]$ выполняют замену переменных:
    \begin{gather*}
      x = \frac{a+b}{2} + \frac{b-a}{2}t,\, t \in [-1; 1] \\
      dx = \frac{b-a}{2}dt \\
      \Rightarrow \int_a^b f(x)\ dx = \frac{b-a}{2} \int_{-1}^{+1} f(\frac{a+b}{2}+\frac{b-a}{2}t)\ dt \\
      \approx \frac{b-a}{2} \sum_{k=1}^S A_k f(\frac{a+b}{2} + \frac{b-a}{2}t_k) \\
      \footnotesize\text{$A_k$ и $t_k$ взяты из справочника для стандартного промежутка}
    \end{gather*}
\end{mainblock}

\end{document}
