\documentclass[a4paper,11pt]{article}

\usepackage{wrapfig}
\usepackage{amsmath}
\usepackage{pgfplots}
\usepackage{xcolor}
\usepackage[most]{tcolorbox}
%Russian-specific packages
%--------------------------------------
\usepackage[T2A]{fontenc}
\usepackage[utf8]{inputenc}
\usepackage[russian, english]{babel}
%--------------------------------------

\definecolor{lemonchiffon}{rgb}{1.0, 0.98, 0.8}
\newtcolorbox{mainblock}{colback=lemonchiffon,grow to right by=-10mm,grow to left by=-10mm,
boxrule=0pt,boxsep=0pt,breakable} % настройки области с изменённым фоном
\definecolor{block-gray}{gray}{0.95} % уровень прозрачности (1 - максимум)
\newtcolorbox{importantblock}{colback=block-gray,grow to right by=-10mm,grow to left by=-10mm,
boxrule=0pt,boxsep=0pt,breakable} % настройки области с изменённым фоном

\makeatletter
\newcommand{\settag}[1]{
  \tag*{(#1)\qquad}
  \edef\@currentlabel{\theequation#1}}
\makeatother

\title{11. Адаптивные квадратурные формулы. Подпрограмма \textbf{QUANC8}}
\author{Андрей Бареков \and Ярослав Пылаев \and По лекциям Устинова С.М.}
\date{\today}

\begin{document}
\maketitle
\newpage

Программа адаптируется к виду функции и использует переменный шаг интегрирования. Маленький там, где функция меняется быстро и обладает большими производными,
  и относительно большой там, где меняется медленно. Программа \textbf{QUANC8} минимизирует объём вычислений при заданных ограничений на погрешность.

\section{Особенности работы QUANC8}
Рассмотрим промежуток длиной $h_i$ внутри промежутка $[a; b]$:
\begin{itemize}
  \item $I_i$ - точное значение интеграла на этом промежутке
  \item $P_i$ - значение интеграла, вычисленное по несоставной квадратурной формуле Ньютона-Котеса с 9 узлами
  \item $Q_i$ - промежуток разделили пополам, на каждой половине применили ту же формулу, результаты сложили
                (по сути, это составная квадратурная формула с вдвое большим количеством узлов).
\end{itemize}
Для составных формул Ньютона-Котеса формула погрешности имеет следующий общий вид:
\[\alpha \frac{(b-a)^{S+1}}{N^S} f^{(S)}(\eta) \settag{*}\]
Для формулы Ньютона-Котеса с 9 узлами $S = 10$. Исходя из формулы погрешности:
\begin{align*}
  (I_i - P_i) & \approx 2^S(I_i - Q_i) \\
  I_i &= \frac{2^SQ_i - P_i}{2^S - 1}
\end{align*}
Тогда:
\[I_i - Q_i = \frac{2^SQ_i - P_i}{2^S - 1} - Q_i = \frac{Q_i - P_i}{2^S - 1} = \underbrace{\frac{Q_i - P_i}{1023}}_{\text{Для QUANC8}}\]
Интеграл на промежутке $h_i$ считается вычисленным, если величина $\Big| \frac{Q_i - P_i}{1023} \Big|$ меньше заданной величины. Требование к погрешности в программе
  реализуется следующим образом:
\[\Bigg| \frac{Q_i - P_i}{1023} \Bigg| \le \frac{h_i}{b-a}\max \{\varepsilon_A, \varepsilon_R \times \widetilde{I}\}\]
$\varepsilon_A$ - абсолютная погрешность, \\
$\varepsilon_R$ - относительная погрешность, \\
$\widetilde{I}$ - грубая оценка интеграла. \\
Пользователь выбирает один из трёх вариантов:
\begin{enumerate}
  \item $\varepsilon_R = 0, \varepsilon_A \ne 0 \Rightarrow$ множитель $\frac{h_i}{b-c}$ отражает вклад в общую погрешность для промежутка.
  \item $\varepsilon_A = 0, \varepsilon_R \ne 0$.
  \item $\varepsilon_A \ne 0, \varepsilon_R \ne 0$. Контролируется смешанная погрешность.
\end{enumerate}

\section{Работа алгоритма}
\begin{enumerate}
  \item Вычисляем $P_i$ и $Q_i$ для всего промежутка $[a; b]$
  \item Если контроль погрешности не прошел, значение функции на правой половине промежутка запоминаем, и повторяем процедуру для левой половины промежутка.
\end{enumerate}
Деление пополам продолжается до тех пор, пока крайний слева промежуток не будет принят. Затем обращаются к ближайшему правому соседу. \\
В программе предусмотрено два ограничения от зацикливания:
\begin{enumerate}
  \item Промежуток разрешается делить пополам не более $30$ раз. Если после $30$-тикратного деления контроль погрешности не прошёл, результат принимается,
        количество промежутков с непрошедшим контролем записывается в целую часть переменной \textbf{FLAG}.
  \item Задаётся максимальное значение вычислений функции $f(x)$. Если эта величина достигнута, то информация о точке $x^*$, в которой программа прервала
        свою работу, записывается в дробную часть переменной \textbf{FLAG}.
\end{enumerate}
Программа имеет параметры:
\begin{equation*}
  (\underbrace{FUN}_{f(x)}, \underbrace{A, B}_{[a; b]}, \underbrace{EA, ER}_{\varepsilon_A, \varepsilon_R}, \underbrace{RESULT}_{\text{знач. интегр.}},
    \underbrace{ERR}_{\text{оценка погр.}}, \underbrace{NOFUN}_{\text{кол-во выч.} f(x)}, FLAG)
\end{equation*}

\end{document}
