\documentclass[a4paper,11pt]{article}

\usepackage{wrapfig}
\usepackage{amsmath}
\usepackage{pgfplots}
\usepackage{xcolor}
\usepackage[most]{tcolorbox}
%Russian-specific packages
%--------------------------------------
\usepackage[T2A]{fontenc}
\usepackage[utf8]{inputenc}
\usepackage[russian, english]{babel}
%--------------------------------------

\definecolor{lemonchiffon}{rgb}{1.0, 0.98, 0.8}
\newtcolorbox{mainblock}{colback=lemonchiffon,grow to right by=-10mm,grow to left by=-10mm,
boxrule=0pt,boxsep=0pt,breakable} % настройки области с изменённым фоном
\definecolor{block-gray}{gray}{0.95} % уровень прозрачности (1 - максимум)
\newtcolorbox{importantblock}{colback=block-gray,grow to right by=-10mm,grow to left by=-10mm,
boxrule=0pt,boxsep=0pt,breakable} % настройки области с изменённым фоном

\makeatletter
\newcommand{\settag}[1]{
  \tag*{(#1)\qquad}
  \edef\@currentlabel{\theequation#1}}
\makeatother

\title{3. Анализ линейных математических моделей, основные его этапы. Фазовые портреты для моделей первого и второго порядка.}
\author{Андрей Бареков \and Ярослав Пылаев \and По лекциям Устинова С.М.}
\date{\today}

\begin{document}
\maketitle
\newpage

\section{Фазовые портреты модели}
\begin{equation}
  \frac{dx}{dt} = f(x),\, x(t_0) = x_0
  \label{eq:DE}
\end{equation}

Для геометрической иллюстрации решений как линейных, так и нелинейных моделей, строят фазовый портрет модели в фазовом пространстве её уравнений
  (компонент вектора $x$) для различных начальных условий. \\
Для уравнений первого порядка имеем следующие виды траекторий: 
\begin{enumerate}
  \item Одноточечная траектория (положение равновесия, $f(x) = 0$)
  \item Интервал (концы интервала - положение равновесия)
  \item Полупрямая (один конец - положение равновесия, другой конец - $\pm \infty$)
\end{enumerate}
% ПРИМЕРЫ И ОЧЕНЬ МНОГО ГРАФИКОВ ДЛЯ ФАЗОВЫХ ПОРТРЕТОВ 1-2 ПОРЯДКА

\section{Анализ линейных моделей}
Обратимся к линейным моделям с постоянной матрицей и отметим основные этапы их анализа:
\[\frac{dx}{dt} = Ax\]
\begin{enumerate}
  \item Получение решения
  \begin{itemize}
    \item Получение стационарного решения $Ax + b = 0$ и анализ его устойчивости \textbf{(*)}
  \end{itemize}
  \item Определение наблюдаемости отдельных составляющих решения, оценка их роли в системе
  \item Оценка чувствительности к параметрам \footnote{На практике интересует чувствительность как к нежелаемым эффектам, так и к параметрам, вводимым специально для решения задач управления}
  \item Решение задачи параметрической идентификации
  \item Решение задач управления или выбора оптимальных значений параметров
\end{enumerate}
\begin{importantblock}
  \textbf{(*)} \\
  \[x^{'} = Ax + b\]
  \[Ax + b = 0,\, Ax = -b \rightarrow \textbf{DECOMP+SOLVE}\]
  Для анализа устойчивости рассчитывают собственные значения матрицы $A$ с помощью $QR$-алгоритма.
  \[\Re(\lambda_k) < 0\]
\end{importantblock}

\end{document}
